\documentclass[11pt]{article}            % Report class in 11 points
\parindent0pt  \parskip10pt             % make block paragraphs
\usepackage{graphicx}
\usepackage{listings}
\graphicspath{ {images/} }
\usepackage{graphicx} %  graphics header file
\begin{document}
\begin{titlepage}
    \centering
  \vfill
    \includegraphics[width=8cm]{uni_logo.png} \\ 
	\vskip2cm
    {\bfseries\Large
	Operating System \\ (CS13217)\\
	
	\vskip2cm
	Lab Report 
	 
	\vskip2cm
	}    

\begin{center}
\begin{tabular}{ l l  } 

Name: Asif Shahzad\\ 
Registration \#: &CSU-F11-110 \\ 
Lab Report \#: & 01 \\ 
 Dated:& 4-30-2018\\ 
Submitted To:& Mr. Usman Ahmed\\ 

 %\hline
\end{tabular}
\end{center}
    \vfill
    The University of Lahore, Islamabad Campus\\
Department of Computer Science \& Information Technology
\end{titlepage}


    
    {\bfseries\Large
\centering
	Experiment \# 1 \\

Introduction to Basic Shell Commands\\
	
	}    
 \vskip1cm
 \textbf {Objective}\\ To conduct our labs on Linux environment and using Ubuntu 12.10 for this purpose
 
 \textbf {Software Tool} \\
 \\Ubuntu 12.10
\\GCC  Compiler


\section{Theory }  
Why Linux            
✓ LINUX is free. \\
✓ The source code of OS can viewed and edit \\
 ✓ It is fully customizable.\\ 
✓ Stability is most Important Feature\\
  ✓  Importance in shared environments and critical applications \\
✓ LINUX a better security structure. \\
✓ High Portability ✓ Easy to port new H/W Platform \\
✓ Written in C which is highly portable \\

\textbf{Layers  in Linux:}\\
Three important parts of Linux are Kernal, Shell and file system. \\


\textbf{Kernel:}\\
                    ✓ Kernel is the heart of the operating system.\\
                     ✓ It is the low level core of the System that is the interface between applications and H/W.\\
                     ✓ Functions  are Manage Memory, I/O devices, allocates the time between user and         Process, inter process communication, sets process priority.\\

\textbf{Shell}:\\
✓ The shell is a program that act as interface between  users and kernel \\
 ✓ It is a command interpreter and also has programming capability of its own.\\

\textbf{File System:}\\
Linux treats everything as a file including hardware devices. Arranged as a directory hierarchy.  \\
✓ The top level directory is known as “root (/)”\\

\textbf{Linux Command:}The following are important one\\
\textbf{HELP}\\
\textbf{ man}[Command name]\\
Purpose: provide all the details about command given. E.g. $ man ls, $ man cat\\
\textbf{Date Time}\\
\textbf{date}\\
\textbf{Purpose}:Prints the system date and time\\
\textbf{Calender}\\
\textbf{Cal}\\
Purpose: Prints an ASCII calendar of the current month




\section{Task}  
\subsection{Procedure: Task 1 }     

\begin{figure*}
\centering


\label{Figure:3}    
\end{figure*}
The minimum number of moves required to solve a Tower of Hanoi puzzle is 2n - 1, where n is the number of disks.

\subsection{Procedure: Task 2 }     

\begin{lstlisting}[language=Python]
import numpy as np
 
def incmatrix(genl1,genl2):
    m = len(genl1)
    n = len(genl2)
    M = None \#to become the incidence matrix
    VT = np.zeros((n*m,1), int)  \#dummy variable
 
    \#compute the bitwise xor matrix
    M1 = bitxormatrix(genl1)
    M2 = np.triu(bitxormatrix(genl2),1) 
 
    for i in range(m-1):
        for j in range(i+1, m):
            [r,c] = np.where(M2 == M1[i,j])
            for k in range(len(r)):
                VT[(i)*n + r[k]] = 1;
                VT[(i)*n + c[k]] = 1;
                VT[(j)*n + r[k]] = 1;
                VT[(j)*n + c[k]] = 1;
 
                if M is None:
                    M = np.copy(VT)
                else:
                    M = np.concatenate((M, VT), 1)
 
                VT = np.zeros((n*m,1), int)
 
    return M
\end{lstlisting}

\section{Conclusion}  


 
\end{document}                          % The required last line